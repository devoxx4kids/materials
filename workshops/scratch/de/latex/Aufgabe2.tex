\section{Spüre die Welt – Bedingte Anweisung und Verzweigung}

\subsection{Mal ein Labyrinth}

Wir brauchen ein Labyrinth als Barriere für unsere Hauptfigur. Wir erzeugen dazu ein weiteres Sprite, welches als Barriere dient.

\begin{enumerate}
\item Klicke auf das \textit{Neues Objekt malen}-Icon 
\item Benutze das Zoom-Tool um komplett herauszuzoomen. Klicke auf die Minus-Lupe.
\item Klicke auf das Rechteck-Symbol und male eine Rechteck um die gesamte Bildfläche.
\item Benutze das Rechteck Tool mit ausgewähltem Rechtfüllmodus, um einige Hindernisse zu malen.
Rechteckfüllmodus deaktivert
\item Klicke auf Ok, um deine Zeichnung abzuschließen.
\item Ändere den Namen des Sprites zu Labyrinth.
\end{enumerate}

\subsection{Abprallen vom Hinderniss}
\begin{enumerate}
\item Klicke auf deine Sprite-Figur.
\item Ziehe jetzt aus dem Steuerung-Panel, das \textit{Wenn grüne Flagge angeklickt} in das Skript-Panel.
\item Füge nun das \textit{wiederhole fortlaufend} hinzu. 
\item Füge nun aus dem Bewegungspanel, \textit{gehe 10er Schritt} in die Lücke.
\item Hänge daran nun aus dem Steuerungs-Panel, die Kachel \textit{falls} hinzu.
\item Wechsle nun in das Fühlen-Panel
\item Wähle nun die \textit{wird berührt}-Kachel, ziehe es ins Skriptpanel in die Wabe der Falls-Kachel und wähle dort das Labyrinth. 
\item Anschließend wählst du aus dem Bewegungspanel \textit{drehe um 90 Grad} und \textit{gehe 10-er Schritte} und ziehst es in die Lücke unter \textit{falls}.
\item Teste dein Programm. Drücke auf die grüne Flagge und starte dein Programm. Beinflusse die Richtung mit Hilfe der Pfeiltasten deiner Tastatur.
\end{enumerate}
